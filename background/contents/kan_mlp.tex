
\begin{frame}{Neural Networks are Function Approximators}

\begin{itemize}
    \item MLPs are inspired by the \strong{Universal Approximation Theorem} \vspace{0.5em}
    \begin{itemize}
        \item Leshno et al. (1993):
        \textit{A standard multilayer FNN can approximate any continuous function to any degree of accuracy iff
        the network’s activation functions are not polynomial}
    \end{itemize}
    \vspace{1em}
    \item Can we be inspired by \strong{KART}?
\end{itemize}

\end{frame}

\begin{frame}{Neural Networks are Function Approximators}

\begin{itemize}
    \item Can we be inspired by \strong{KART}?
    \begin{itemize}
        \item KART claims that learning a high-dimensional function reduces to learning a polynomial number of 1D functions then, why \textbf{inspired}? \dots
        \item \dots these 1D functions can be non-smooth (aka \textit{not learnable!})
        \item \dots using only 2-layer nonlinearities and $2n + 1$ terms in the hidden layer is pretty restrictive
    \end{itemize}
    \vspace{1em}
    \item \textit{But},
    \begin{itemize}
        \item We can argue that most ML-concerning functions are smooth, and
        \item We can just stack $\phi$ layers and pray
    \end{itemize}
    \vspace{1em}
    \item So, we are no longer using KA(R)T but moving towards a KA(A)T
    \vspace{1em}
    \item This is how \strong{Kolmogorov-Arnold Networks (KANs)} are born (Ziming Liu et al., 2024)
\end{itemize}

\end{frame}
